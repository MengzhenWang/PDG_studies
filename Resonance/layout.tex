\section{Layout}

\begin{enumerate}

\item Unnecessary blank space should be avoided, between paragraphs or
  around figures and tables.

\item Figure and table captions should be concise and use a somewhat smaller typeface
  than the main text, to help distinguish them. This is achieved by 
  inserting \verb!\small! at the beginning of the caption.
  (NB with the latest version of the file \verb!preamble.tex! this is automatic)
  Figure captions go below the figure, table captions go above the
  table.

\item Captions and footnotes should be punctuated correctly, like
  normal text. The use of too many footnotes should be avoided:
  typically they are used for giving commercial details of companies,
  or standard items like coordinate system definition or the implicit
  inclusion of charge-conjugate processes.\footnote{If placed at the end
    of a sentence, the footnote symbol normally follows the
    punctuation; if placed in the middle of an equation, take care to
    avoid any possible confusion with an index.}$^,$\footnote{The standard footnote reads: ``The inclusion of charge-conjugate processes is implied
    throughout.'' This may need to be modified, for example with ``except in the discussion of asymmetries.''}

\item Tables should be formatted in a simple fashion, without
  excessive use of horizontal and vertical lines. See
  Table~\ref{tab:example} for an example.

\item Figures and tables should normally be placed so that they appear
  on the same page as their first reference, but at the top or bottom
  of the page; if this is not possible, they should come as soon as
  possible afterwards.  They must all be referred to from the text.

\item If one or more equations are referenced, all equations should be numbered using parentheses as shown in
  Eq.~\ref{eq:CKM},
  \begin{equation}
    \label{eq:CKM}
    V_{\uquark\squark}V_{\uquark\bquark}^* + 
    V_{\cquark\squark}V_{\cquark\bquark}^* + 
    V_{\tquark\squark}V_{\tquark\bquark}^* = 0 \ . 
  \end{equation}
  
\item Displayed results like
  \begin{equation*}
    \BF(\decay{\Bs}{\mumu}) < 1.5 \times 10^{-8} \text{ at 95\% CL}
  \end{equation*}
  should in general not be numbered.

\item Numbered equations should be avoided in captions and footnotes.

\item Displayed equations are part of the normal grammar of the
  text. This means that the equation should end in full stop or comma if
  required when reading aloud. The line after the equation should only
  be indented if it starts a new paragraph.

\item Sub-sectioning should not be excessive: sections with more than three
levels of index (1.1.1) should be avoided.

%\item It is generally preferable to itemize a list using numbers rather
%than bullets.

\item Acronyms should be defined the first time they are used,
  \eg ``Monte Carlo~(MC) events containing a doubly
  Cabibbo-suppressed~(DCS) decay have been generated.''
  The abbreviated words should not be capitalised if it is not naturally
  written with capitals, \eg quantum chromodynamics (QCD),
  impact parameter (IP), boosted decision tree (BDT).
  Avoid acronyms if they are used three times or less.
  A sentence should never start with an acronym and its better to
  avoid it as the last word of a sentence as well.

\end{enumerate}

\begin{table}[t]
  \caption{
    %\small %captions should be a little bit smaller than main text
    Background-to-signal ratio estimated in a $\pm 50\mevcc$ 
    mass window for the prompt and long-lived backgrounds, and the 
    minimum bias rate.}
\begin{center}\begin{tabular}{lccc}
    \hline
    Channel                           & $B_{\mathrm{pr}}/S$ & $B_{\mathrm{LL}}/S$   & MB rate       \\ 
    \hline
    \BsToJPsiPhi              & $ 1.6 \pm 0.6$ & $ 0.51 \pm 0.08$ & $\sim 0.3$ Hz \\
    \BdToJPsiKst              & $ 5.2 \pm 0.3$ & $1.53 \pm 0.08 $ & $\sim 8.1$ Hz \\
    \decay{\Bp}{\jpsi\Kstarp} & $ 1.6 \pm 0.2$ & $0.29 \pm 0.06$  & $\sim 1.4$ Hz \\
    \hline
  \end{tabular}\end{center}
\label{tab:example}
\end{table}

% old table with vertical lines
%\begin{table}[t]
%  \caption{
%    \small %captions should be a little bit smaller than main text
%    Background-to-signal ratio estimated in a $\pm 50\mevcc$ 
%    mass window for the prompt and long-lived backgrounds, and the 
%    minimum bias rate.}
%\begin{center}\begin{tabular}{l|c|c|c}
%    Channel                           & $B_{\mathrm{pr}}}/S$ & $B_{{\mathrm{LL}}/S$   & MB rate       \\ 
%    \hline
%    \BsToJPsiPhi              & $ 1.6 \pm 0.6$ & $ 0.51 \pm 0.08$ & $\sim 0.3$ Hz \\
%    \BdToJPsiKst              & $ 5.2 \pm 0.3$ & $1.53 \pm 0.08 $ & $\sim 8.1$ Hz \\
%    \decay{\Bp}{\jpsi\Kstarp} & $ 1.6 \pm 0.2$ & $0.29 \pm 0.06$  & $\sim 1.4$ Hz \\
%  \end{tabular}\end{center}
%\label{tab:example}
%\end{table}
